\section{Commandes de base, créer un commit}

\usebackgroundtemplate{\includegraphics[width=\paperwidth,height=\paperheight]{background-light}}
\begin{frame}
  \tableofcontents[sectionstyle=show/shaded]
\end{frame}
\usebackgroundtemplate{}

\begin{frame}[fragile]{Créer le dépôt Git}
  Pour un nouveau projet :
\begin{footnotesize}
\begin{verbatim}
$ mkdir project
$ cd project/
$ git init
\end{verbatim}
\end{footnotesize}

  \pause
  \bigskip
  Pour un projet existant :
\begin{footnotesize}
\begin{verbatim}
$ git clone git://example.net/project
$ cd project/
\end{verbatim}
\end{footnotesize}

  \pause
  \bigskip
  Répertoire caché \texttt{.git/} (unique) :
\begin{footnotesize}
\begin{verbatim}
$ ls .git/
config  description  HEAD  hooks/  info/  objects/  refs/
\end{verbatim}
\end{footnotesize}
\end{frame}

\begin{frame}{États d'un fichier}
  \textit{\textbf{Untracked}}
  \begin{itemize}
    \item non pris en compte par Git
  \end{itemize}

  \pause
  \medskip
  \textit{\textbf{Unmodified/Committed}}
  \begin{itemize}
    \item aucune modification
  \end{itemize}

  \pause
  \medskip
  \textit{\textbf{Modified}}
  \begin{itemize}
    \item fichier modifié
    \item pas pris en compte pour le prochain commit
  \end{itemize}

  \pause
  \medskip
  \textit{\textbf{Staged}}
  \begin{itemize}
    \item fichier ajouté, modifié, supprimé ou déplacé
    \item pris en compte pour le prochain commit
  \end{itemize}
\end{frame}

\begin{frame}{États d'un fichier}
\begin{center}
\begin{tikzpicture}[thick, node distance=40mm]
  \node [myrect] (wd) {Working directory};
  \node [myrect, below=10mm of wd, align=center, fill=red] (uu) {Untracked \\ Modified};
  \node [myrect, right of=wd] (staging) {Staging area};
  \node [myrect, right of=uu, align=center, fill=green] (staged) {Staged};
  \node [myrect, right of=staging] (repository) {Repository};
  \node [myrect, right of=staged, fill=yellow] (committed) {Committed};
  \draw [arrow] (staged) -- (uu) node [above,midway] {git add};
  \draw [arrow] (committed) -- (staged) node [above,midway] {git commit};
\end{tikzpicture}
\end{center}
\end{frame}

\begin{frame}[fragile]{Créer un nouveau fichier}
\begin{footnotesize}
\begin{verbatim}
$ echo hello > README
\end{verbatim}
\end{footnotesize}

État du fichier : \textit{\textbf{untracked}}

\pause
\begin{footnotesize}
\begin{alltt}
$ git status
# On branch master
#
# Initial commit
#
# \textcolor{ansired}{Untracked files}:
#   (use "git add <file>..." to include in what will be committed)
#
#       \textcolor{ansired}{README}
nothing added to commit but untracked files present
\end{alltt}
\end{footnotesize}
\end{frame}

\begin{frame}[fragile]{Créer un nouveau fichier}
\begin{footnotesize}
\begin{verbatim}
$ git add README
\end{verbatim}
\end{footnotesize}

État du fichier : \textit{\textbf{untracked}} $ \longrightarrow $ \textit{\textbf{staged}}

\pause
\begin{footnotesize}
\begin{alltt}
$ git status
# On branch master
#
# Initial commit
#
# Changes to be committed:
#   (use "git rm --cached <file>..." to unstage)
#
#	      \textcolor{ansigreen}{new file:   README}
#
\end{alltt}
\end{footnotesize}
\end{frame}

\begin{frame}[fragile]{Créer le commit}
\begin{footnotesize}
\begin{verbatim}
$ git commit
[écrire le message du commit]
\end{verbatim}

\pause
\begin{verbatim}
$ git log
commit c3aab8bb6cca644162a4fa82df283682717da3d4
Author: Your Name <foo@bar.be>
Date:   Wed Feb 1 15:19:03 2012 +0100

    Titre du commit (pas trop long)

    Plus longue description.
    Ligne vide après le titre.
\end{verbatim}
\end{footnotesize}
\end{frame}

\begin{frame}[fragile]{Modifier un fichier}
État du fichier README : \textit{\textbf{unmodified}}

\pause
\begin{footnotesize}
\begin{verbatim}
$ echo world >> README
\end{verbatim}
\end{footnotesize}

État : \textit{\textbf{modified}}

\pause
\begin{footnotesize}
\begin{verbatim}
$ git add README
\end{verbatim}
\end{footnotesize}

État : \textit{\textbf{staged}}

\pause
\begin{footnotesize}
\begin{verbatim}
$ git commit
\end{verbatim}
\end{footnotesize}

État : \textit{\textbf{committed}}
\end{frame}

\begin{frame}[fragile]{Diff}
  Voir les modifications avant de créer un commit.
\begin{small}
\begin{alltt}
$ echo new-text > README

$ git diff
diff --git a/README b/README
index 2e85c45..4320c6f 100644
--- a/README
+++ b/README
\textcolor{ansicyan}{@@ -1,2 +1 @@}
\textcolor{ansired}{-hello
-world}
\textcolor{ansigreen}{+new-text}
\end{alltt}
\end{small}
\end{frame}

\begin{frame}[fragile]{Aide}
La liste des commandes :
\begin{small}
\begin{verbatim}
$ git help
\end{verbatim}
\end{small}

\bigskip
Page de manuel d'une commande :
\begin{small}
\begin{verbatim}
$ git help <cmd>
$ git help <cmd> --web
$ git <cmd> --help
\end{verbatim}
\end{small}
\end{frame}

\begin{frame}{Résumé des commandes}
  \begin{ttfamily}
    \begin{itemize}
      \item git init
      \item git clone
      \item git status
      \item git add <file>
      \item git rm/mv <file>
      \item git commit
      \item git log
      \item git diff
      \item git help
    \end{itemize}
  \end{ttfamily}
\end{frame}
