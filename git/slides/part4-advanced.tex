\section{Autres fonctionnalités}

\usebackgroundtemplate{\includegraphics[width=\paperwidth,height=\paperheight]{background-light}}
\begin{frame}
  \tableofcontents[sectionstyle=show/shaded]
\end{frame}
\usebackgroundtemplate{}

\begin{frame}{\code{git stash}}
\begin{center}
\begin{tikzpicture}[thick,commit2/.style={commit, node distance=11mm}]
  \node [commit2] (98ca9) {98ca9};
  \node [commit2, above of=98ca9] (34ac2) {34ac2};
  \draw [arrow] (98ca9) -- (34ac2);
  \node [commit2, above of=34ac2] (f30ab) {f30ab};
  \draw [arrow] (34ac2) -- (f30ab);
  \node [commit2, above of=f30ab] (c2b9e) {c2b9e};
  \draw [arrow] (f30ab) -- (c2b9e);
  \node [commit2, above of=c2b9e] (a071e) {a071e};
  \draw [arrow] (c2b9e) -- (a071e);

  \node [commit, right=25mm of c2b9e] (stash0) {stash@\{0\}};
  \node [commit, below of=stash0] (stash1) {stash@\{1\}};
  \node [commit, below of=stash1] (stash2) {stash@\{2\}};
  \node [commit, below of=stash2] (stash3) {...};
  \begin{pgfonlayer}{background}
    \node [head, fit=(stash0) (stash3)] {};
  \end{pgfonlayer}

  \draw [arrow, dashed, opacity=.5] (c2b9e) -- (stash0);
  \draw [arrow, dashed, opacity=.5] (c2b9e) -- (stash1);
  \draw [arrow, dashed, opacity=.5] (98ca9) -- (stash2);
\end{tikzpicture}
\end{center}
\end{frame}

\begin{frame}{\code{git stash}}
  git stash \begin{itemize}
    \item save [message]
    \pause
    \item list
    \item show [stash]
    \pause
    \item apply [stash]
    \item pop [stash]
  \end{itemize}
\end{frame}

\begin{frame}[fragile]{\code{git stash save}}
\begin{small}
\begin{alltt}
$ git status
# On branch master
# Changes not staged for commit:
#
#       \textcolor{ansired}{modified:   main.c}
#
no changes added to commit (use "git add"/"git commit -a")

$ git stash save
Saved working directory and index state WIP on master: 8540fea
HEAD is now at 8540fea message

$ git status
# On branch master
nothing to commit (working directory clean)
\end{alltt}
\end{small}
\end{frame}

\begin{frame}[fragile]{\code{git stash list, show}}
\begin{alltt}
$ git stash list
stash@\{0\}: WIP on master: 8540fea message

$ git stash show -p
\textcolor{ansired}{diff –git a/main.c b/main.c
index f2ad6c7..2f773ae 100644
--- a/main.c
+++ b/main.c}
\textcolor{ansicyan}{@@ -1,3 +1,5 @@}
\textcolor{ansigreen}{+#include <stdio.h>
+}
 int main() \{
    printf("Hello, world!");
    return 0;
 \}
\end{alltt}
\end{frame}

\begin{frame}{Commandes avancées}
  \textbf{git grep}
  \begin{itemize}
    \item grep(1) sur les fichiers pris en compte par git
  \end{itemize}

  \bigskip
  \textbf{git tag}
  \begin{itemize}
    \item crée un \emph{tag}, une référence fixe vers un commit
  \end{itemize}

  \bigskip
  \textbf{git revert}
  \begin{itemize}
    \item crée un \emph{commit} qui annule un autre
  \end{itemize}
\end{frame}

\begin{frame}[fragile]{\code{git blame}}
montre qui a modifié le fichier, ligne par ligne

\begin{footnotesize}
\begin{verbatim}
$ git blame git.c
85023 (Junio C Hamano      2006-12-19  1) #include "builtin.h"
2b11e (Johannes Schindelin 2006-06-05  2) #include "cache.h"
fd5c3 (Thiago Farina       2010-08-31  3) #include "exec_cmd.h"
fd5c3 (Thiago Farina       2010-08-31  4) #include "help.h"
575ba (Matthias Lederhofer 2006-06-25  5) #include "quote.h"
d8e96 (Jeff King           2009-01-28  6) #include "run-command.h"
8e49d (Andreas Ericsson    2005-11-16  7)
822a7 (Ramsay Allan Jones  2006-07-30  8) const char git_usage_string[] =
f2dd8 (Jon Seymour         2011-05-01  9)        "git [--version] [--exec-path[=<path>]] [--html-path] [--man-path] [--info-path]\n"
\end{verbatim}
\end{footnotesize}
\end{frame}

\begin{frame}[fragile]{\code{git reset}}
Change le pointeur de la branche courante
\begin{footnotesize}
\begin{alltt}
$ git reset HEAD^ # recule la branche d'un commit
Unstaged changes after reset:
M      README

$ git status
# On branch master
# Changes not staged for commit:
#
#       \textcolor{ansired}{modified:   README}
\end{alltt}
\begin{center}
\begin{tikzpicture}[thick, node distance=25mm]
  \node [align=center] (command) at (2,2.3) {\code{\$ git reset \code{-}-hard 98ca9}};
  \node [commit] (98ca9) {98ca9};
  \node [commit, right of=98ca9] (34ac2) {34ac2};
  \draw [arrow] (98ca9) -- (34ac2);
  \node [commit, right of=34ac2] (f30ab) {f30ab};
  \draw [arrow] (34ac2) -- (f30ab);

  \node [branch, above=6mm of f30ab, opacity=.5] (old_master) {master};
  \draw [arrow, opacity=.5] (f30ab) -- (old_master);

  \node [branch, above=6mm of 98ca9] (master) {master};
  \draw [arrow] (98ca9) -- (master);
\end{tikzpicture}
\end{center}
\end{footnotesize}
\end{frame}

\begin{frame}[fragile]{\code{git commit \code{-}-amend}}
Pour modifier le dernier \emph{commit} :
\begin{itemize}
  \item ajouter une modification ;
  \item modifier le message du \emph{commit}.
\end{itemize}

\bigskip
\begin{verbatim}
$ edit some_file
$ git add some_file
$ git commit --amend
\end{verbatim}
\end{frame}

\begin{frame}[fragile]{\code{git rebase \code{-}-interactive}}
Permet de modifier l'historique
\begin{footnotesize}
\begin{verbatim}
$ git rebase -i HEAD~3

pick 4eeebe5 bulk-checkin: allow the same data to be multiply hashed
pick 127b177 bulk-checkin: support chunked-object encoding
pick 973951a chunked-object: fallback checkout codepaths

# Rebase 617312b..973951a onto 617312b
#
# Commands:
#  p, pick = use commit
#  r, reword = use commit, but edit the commit message
#  e, edit = use commit, but stop for amending
#  s, squash = use commit, but meld into previous commit
#  f, fixup = like "squash", but discard this commit's log message
#  x, exec = run command (the rest of the line) using shell
#
# If you remove a line here THAT COMMIT WILL BE LOST.
# However, if you remove everything, the rebase will be aborted.
#
\end{verbatim}
\end{footnotesize}
\end{frame}

\begin{frame}[fragile]{\code{git reflog}}
Log des opérations sur les \emph{commits}
\begin{footnotesize}
\begin{verbatim}
$ git reflog
b0d66 HEAD@{0}: commit (amend): add a setting to require a filter to be successful
97395 HEAD@{1}: checkout: moving from master to man
b0d66 HEAD@{2}: rebase: aborting
9cda8 HEAD@{3}: rebase -i (pick): grep: drop grep_buffer's "name" parameter
b9ef9 HEAD@{4}: rebase -i (pick): convert git-grep to use grep_source interface
837de HEAD@{5}: rebase -i (pick): grep: make locking flag global
84f3d HEAD@{6}: checkout: moving from master to 84f3d
b0d66 HEAD@{7}: pull: Fast-forward
f6b50 HEAD@{8}: cherry-pick: add a TODO
98c05 HEAD@{9}: reset: moving to HEAD^
e11ee HEAD@{10}: checkout: moving from master to pu
77eac HEAD@{11}: commit: add a TODO
75f43 HEAD@{12}: commit: use the correct format identifier for unsigned long: %lu
f88cc HEAD@{13}: pull --rebase: git grep
07873 HEAD@{14}: pull : Fast-forward
f70f7 HEAD@{15}: clone: from git://git.kernel.org/pub/scm/git/git.git
\end{verbatim}
\end{footnotesize}
\end{frame}

\begin{frame}[fragile]{\code{git add \code{-}-patch}}
Permet d'ajouter une partie des modifications d'un fichier
\begin{footnotesize}
\begin{alltt}
$ git add -p
diff --git a/README b/README
\textcolor{ansicyan}{@@ -1,5 +1,7 @@}
 1
\textcolor{ansigreen}{+2}
 3
\textcolor{ansigreen}{+4}
\textcolor{ansiblue}{Stage this hunk [y,n,q,a,d,/,s,e,?]?} s
Split into 2 hunks.
\textcolor{ansicyan}{@@ -1,2 +1,3 @@}
 1
\textcolor{ansigreen}{+2}
 3
\textcolor{ansiblue}{Stage this hunk [y,n,q,a,d,/,j,J,g,e,?]?} n
\textcolor{ansicyan}{@@ -2,4 +3,5 @@}
 3
\textcolor{ansigreen}{+4}
\textcolor{ansiblue}{Stage this hunk [y,n,q,a,d,/,K,g,e,?]?} y
\end{alltt}
\end{footnotesize}
\end{frame}

\begin{frame}{git cherry-pick}
Applique un \emph{commit} sur la branche courante.

\begin{center}
\begin{tikzpicture}[thick, node distance=25mm]
  \node [align=center] (command) at (2,2.5) {\code{\$ git cherry-pick 4ce4a}};
  \node [commit] (98ca9) {98ca9};
  \node [commit, right of=98ca9] (34ac2) {34ac2};
  \draw [arrow] (98ca9) -- (34ac2);
  \node [commit, right of=34ac2] (f30ab) {f30ab};
  \draw [arrow] (34ac2) -- (f30ab);

  \node [commit, below=8mm of 98ca9] (bf7e0) {bf7e0};
  \node [commit, right of=bf7e0] (4ce4a) {4ce4a};
  \draw [arrow] (bf7e0) -- (4ce4a);
  \node [commit, right of=4ce4a] (e73f0) {e73f0};
  \draw [arrow] (4ce4a) -- (e73f0);
  \node [commit, right of=e73f0] (ffbcb) {ffbcb};
  \draw [arrow] (e73f0) -- (ffbcb);

  \node [branch, above=6mm of 34ac2, opacity=.5] (old_master) {master};
  \draw [arrow, opacity=.5] (34ac2) -- (old_master);

  \node [branch, above=6mm of f30ab] (master) {master};
  \draw [arrow] (f30ab) -- (master);

  \node [branch, below=6mm of ffbcb] (feature) {feature};
  \draw [arrow] (ffbcb) -- (feature);

  \draw [arrow, dashed, opacity=.5] (f30ab) -- (4ce4a);
\end{tikzpicture}
\end{center}
\end{frame}

\begin{frame}[fragile]{\code{git bisect}}
Permet de trouver l'origine d'un bug

\begin{footnotesize}
\begin{verbatim}
$ git bisect start bad_commit good_commit
Bisecting: 20 revisions left to test after this (roughly 4 steps)
[e4f3edc] Sync with maint

$ git bisect run ./mytests
Bisecting: 9 revisions left to test after this (roughly 3 steps)
[42226d] pack-objects: remove bogus comment
...
Bisecting: 0 revisions left to test after this (roughly 1 step)
[9a4749] checkout -m: no need to insist on having all 3 stages

8280f2e0a0f4776c4b3008c9172fc0a3e7361534 is the first bad commit
commit 8280f2e0a0f4776c4b3008c9172fc0a3e7361534
Author: eregon <mail@me>
Date:   Sat Feb 18 17:36:47 2012 +0100

    Noooo! You found my hidden bug!
\end{verbatim}
\end{footnotesize}
\end{frame}

% expain briefly internals: sha1, parents, ...

\begin{frame}{Services d'hébergement}
  \begin{itemize}
    \item GitHub
    \medskip
    \item Gitorious
    \medskip
    \item Bitbucket, assembla (dépôts privés gratuits)
    \medskip
    \item En INGI (voir le wiki)
          \url{http://wiki.student.info.ucl.ac.be/index.php/Git}
  \end{itemize}
\end{frame}

\usebackgroundtemplate{\includegraphics[width=\paperwidth,height=\paperheight]{background-light}}
\begin{frame}{Liens}
\begin{itemize}
  \item Les manpages: \code{git help [-\code{-}web] <cmd>}
  \item \url{http://git-scm.com/book}: Pro Git book
  \item \url{http://louvilug.tuxfamily.org/}: Slides, aide-mémoire, exercices
\end{itemize}
\end{frame}
\usebackgroundtemplate{}
