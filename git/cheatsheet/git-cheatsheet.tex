\documentclass[a4paper,landscape,10pt]{article}
\usepackage[twocolumn,columnsep=20mm,hmargin={15mm,15mm},vmargin={15mm,15mm}]{geometry}
\usepackage{nopageno}

\setlength{\parindent}{0pt}

\usepackage[utf8]{inputenc}
\usepackage[french]{babel}
\usepackage[T1]{fontenc}
\usepackage{url}

\usepackage{listings}
\lstset{
  language=bash,
  basicstyle=\ttfamily,
  showstringspaces=false,
}

\newcommand{\code}[1]{\lstinline{#1}}
\newcommand{\shell}[1]{\lstinline{\$ #1}}

\begin{document}

\begin{center}
  {\huge Git : cheatsheet}
\end{center}

\section{Configuration globale}

\begin{tabular}{ll}
  Nom     & \shell{git config --global user.name "Your Name"} \\
  Email   & \shell{git config --global user.email "foo@bar.be"} \\
  Couleur & \shell{git config --global color.ui auto} \\
\end{tabular}

\section{Création du dépôt}

Dans le dossier principal du projet:

\shell{git init}

\medskip
Ou bien faire un clone d'un projet existant :

\shell{git clone <url>}

\section{Commandes de base}

Voir l'état actuel du dépôt :

\shell{git status}

\medskip
Voir l'historique :

\shell{git log [-p]}

\shell{git lola}

\medskip
Voir les modifications en cours :

\shell{git diff}

\section{Création d'un \emph{commit}}

\subsection*{Marquer ce qu'il faut mettre dans le \emph{commit}}

En fonction du type de modification, il faut utiliser la commande appropriée.

\begin{tabular}{ll}
  Modifier & \shell{git add <file>}\textsuperscript{1} \\
  Ajouter & \shell{git add <file>} \\
  Supprimer & \shell{git rm <file>} \\
  Déplacer & \shell{git mv <old_name> <new_name>} \\[1em]
\end{tabular}

\textsuperscript{1} Si on veux marquer toutes les modifications (mais pas les nouveau fichiers),\\
\hspace*{2mm} il y a un raccourci: \shell{git commit -a}.

\subsection*{Écrire le message du commit et le créer}

\shell{git commit}\\

L'option \code{-m} peut être utilisée pour passer le message directement:\\
\shell{git commit -m "message du commit"}

\section{Branches}

\subsection*{Créer une branche}

\shell{git branch <name>}

\subsection*{Changer de branche}

\shell{git checkout <branch>}

\subsection*{\emph{Merge} une branche}

\emph{Merge} la branche courante avec \code{other_branch}.

\shell{git merge <other_branch>}

\subsection*{\emph{Rebase} une branche}

\emph{Rebase} la branche courante «~au dessus~» de \code{other_branch}.

\shell{git checkout <other_branch>}

\subsection*{Prendre les modifications distantes}

\shell{git pull}

\subsection*{Envoyer ses modifications}

\shell{git push}

\end{document}
