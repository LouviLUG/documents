\documentclass[a5paper,11pt]{article}
\usepackage[T1]{fontenc}
\usepackage[utf8]{inputenc}
\usepackage[francais]{babel}
\usepackage{lmodern}
\usepackage{graphicx}
\usepackage{url}

% marges de la page
\usepackage[top=0.6cm, bottom=0.6cm, left=1cm, right=1cm]{geometry}

% propriétés du fichier PDF et liens
\usepackage{hyperref}
\hypersetup{
  pdfauthor   = {Sébastien Wilmet},
  pdftitle    = {LaTeXila (Louvain-li-Nux)},
  pdfcreator  = {Texlive},
  pdfproducer = {Texlive},
  colorlinks  = true,
  linkcolor   = black,
  citecolor   = black,
  urlcolor    = black
}

% titres personnalisés
\def\titre#1#2{
  \noindent
  \begin{minipage}{0.14\linewidth}
  \includegraphics[height=1.7cm]{#1}
  \end{minipage}
  \begin{minipage}{0.85\linewidth}
    {\LARGE #2}

    \begin{flushright}
      \url{https://wiki.gnome.org/Apps/LaTeXila}
    \end{flushright}
  \end{minipage}

  \vspace{0.5cm}
}

\pagestyle{empty}

\begin{document}

\titre{latexila.png}{\LaTeX{}ila, un éditeur \LaTeX{} pour GNOME}

\LaTeX{} permet d'écrire des documents de toutes sortes : des articles, rapports, livres, transparents, présentations, lettres, mémoires, syllabus, … Les applications telles que LibreOffice, Abiword ou encore Microsoft Word sont de type WYSIWYG («~\textit{What you see is what you get}~»). \LaTeX{}, par contre, est un langage informatique à balises. Par exemple, la balise \texttt{\textbackslash{}tableofcontents} permet d'afficher la table des matières, qui est générée automatiquement. Les fichiers sources sont compilés pour donner généralement un fichier PDF.

Pour écrire un document en \LaTeX{}, trois logiciels sont nécessaires (pour simplifier) : un éditeur de texte, pour éditer les fichiers sources ; le compilateur qui génère le fichier PDF ; et un visionneur de documents PDF. Bien qu'un simple éditeur de texte suffit, il est plus simple d'utiliser un éditeur \LaTeX{}, qui possède des fonctionnalités propres à \LaTeX{}.

GNOME est un gestionnaire de bureaux disponible sous GNU/Linux. \LaTeX{}ila est un éditeur \LaTeX{} qui s'intègre bien à GNOME, car la bibliothèque graphique GTK+ est utilisée. C'est un logiciel libre, sous licence GPL~3, et est écrit en langage Vala, qui est de plus en plus utilisé pour développer des applications en GTK+.

Comme fonctionnalités principales, il y a :
\begin{itemize}
  \item La complétion des balises : quand on tape le début d'une balise, des propositions apparaissent ;
  \item Compiler, convertir et visionner un document en un clic, avec l'affichage des erreurs éventuelles. Ceci peut être configuré assez finement ;
  \item L'affichage de la structure du document, c'est-à-dire une liste arborescente des chapitres, sections, sous-sections, etc. ainsi que d'autres éléments comme les figures et les tableaux. Cliquer sur un élément permet d'aller à l'endroit de celui-ci dans le fichier source ;
  \item Des tableaux de symboles scientifiques. En cliquant sur un symbole, la balise correspondante est insérée dans le document ;
  \item La correction orthographique ;
  \item La gestion de modèles, qui sert de base à la création de nouveaux documents : quelques modèles de base sont disponibles pour par exemple créer un article ou un rapport, mais l'utilisateur peut créer ses propres modèles ;
  \item Quand un document est divisé en plusieurs fichiers sources, il est utile de créer un projet. La création d'un projet est très simple : il suffit de spécifier le dossier ainsi que le fichier principal.
\end{itemize}

\end{document}
