\documentclass[a5paper,11pt]{article}
\usepackage[T1]{fontenc}
\usepackage[utf8]{inputenc}
\usepackage[francais]{babel}
\usepackage{lmodern}
\usepackage{graphicx}
\usepackage{url}

% marges de la page
\usepackage[top=0.6cm, bottom=1cm, left=1cm, right=1cm]{geometry}

% propriétés du fichier PDF et liens
\usepackage{hyperref}
\hypersetup{
  pdfauthor   = {Sébastien Wilmet},
  pdftitle    = {Distributions GNU/Linux (LouviLUG)},
  pdfcreator  = {Texlive},
  pdfproducer = {Texlive},
  colorlinks  = true,
  linkcolor   = black,
  citecolor   = black,
  urlcolor    = black
}

% titres personnalisés
\def\titre#1#2{
  \noindent
  \begin{minipage}{0.14\linewidth}
  \includegraphics[height=1.7cm]{#1}
  \end{minipage}
  \begin{minipage}{0.85\linewidth}
    {\LARGE #2}

    \begin{flushright}
      LouviLUG, le 9/11/2011 (\url{louvilug.be}).
    \end{flushright}
  \end{minipage}

  \vspace{0.5cm}
}

% en-tête et pied de page
\usepackage{fancyhdr}
\lhead{}
\chead{}
% dirty hack…
\rhead{
  \vspace{1.1cm}
  \hspace*{0.4cm}
  \thepage/\pageref{last-page}
  \hspace*{-0.4cm}
  \vspace{-1.1cm}
}
\renewcommand{\headrulewidth}{0pt}
\lfoot{}
\cfoot{}
\rfoot{}
\pagestyle{fancy}

\begin{document}

\titre{tux.pdf}{Distributions GNU/Linux~: introduction}

Vous avez donc déjà entendu parler de Linux, bien~! Mais il se fait que Linux a
un bon ami qui s'appelle GNU. Ensemble, ils forment un système d'exploitation
complet, comme Windows ou Mac~OS~X, mais la grosse différence est que c'est un
OS (\textit{Operating System}) \textbf{libre}.\footnote{Voir
\url{http://fr.wikipedia.org/wiki/Logiciel_libre}.}

\bigskip Il existe plusieurs \textit{distributions} GNU/Linux. Ce sont en
quelque sorte des variantes, mais qui se basent toutes plus ou moins sur un même
ensemble de logiciels. Quels sont les points principaux qui les différencient~?

\bigskip La \textbf{facilité d'installation}. Certaines distrib' offrent plus de
choix sur ce que l'on veut installer, ce qui rajoute une certaine complexité.
D'autres visent une installation la plus simple et la plus rapide possible, qui
convient à la plupart des gens normaux.

\bigskip La \textbf{gestion des paquets} est souvent différente. Oui, pour
installer un programme sur Linux, on utilise un système très pratique, qui
s'appelle un \textit{gestionnaire de paquets}. Pas besoin d'aller télécharger au
fin fond du web un fichier suspect où il faut cliquer sans arrêt sur «~suivant~»
et puis de devoir redémarrer l'ordinateur pour on ne sait quelle raison. Non, à
la place, il suffit de faire une gentille recherche pour trouver le nom du
paquet à installer, et ensuite tout se fait pratiquement tout seul. L'ensemble
des logiciels sont donc installés et mis à jour de façon uniforme~!

\bigskip  \textbf{La fréquence des sorties de nouvelles versions}, ainsi que la
\textbf{durée de support}. Une nouvelle version d'Ubuntu et de Fedora sort
tous les six mois. Chez Debian, ça sort «~quand c'est prêt~» (environ tous les
deux ans). D'autres distributions comme Gentoo et Arch Linux sont en
\textit{rolling release}~: c'est continuellement mis à jour.

\bigskip  Sans oublier la qualité et la quantité de \textbf{documentation},
ainsi que la \textbf{communauté} (forums, \ldots), mais surtout
l'\textbf{idéologie}. Certaines distrib' attachent plus d'importance aux
libertés fondamentales, d'autres se veulent universelles et veulent offrir un
maximum de choix, d'autres encore se veulent à la pointe de la technologie, ou
au contraire préfèrent une stabilité maximale. Ubuntu est généralement admis
comme un bon point de départ, mais peut-être qu'il existe une chaussure mieux
taillée à votre pied~!

%%%%%%%%%%%%%%%%%%%%%%%%%%%%%%%%%%%%%%%%%%%%%%%%%%%%%%%%%%%%%%%%%%%%%%%%%%%%%%%
\pagebreak \titre{debian.pdf}{Distributions GNU/Linux~: Debian}

Nous commençons par Debian, car Ubuntu en est un dérivé direct.

\bigskip Debian a été créée en 1993, ce qui en fait une des plus vieilles
distributions encore en activité. Le développement de Debian se fait de manière
vraiment \textbf{communautaire}. Ce n'est donc pas une distribution commerciale. Debian se définit comme «~Le système d'exploitation \textbf{universel}~» : on peut l'installer sur un très grand nombre de matériel électronique différent (ordinateurs, GSM, routeurs, …), et il existe un très grand nombre de paquets.

\bigskip Une chose importante dans Debian est sa philosophie, notamment son
fameux \textbf{contrat social} qui met l'accent sur le respect des principes du
logiciel libre, auxquels les contributeurs adhèrent.

\bigskip Une nouvelle version de Debian sort environ tous les deux ans («~quand
c'est prêt~»), se veut très stable et est supportée pendant longtemps (plusieurs
années). En plus, chaque version porte le nom d'un des personnages de
\textbf{Toy Story}~! {\scriptsize (Que demander de plus~?)} La version stable
est bien adaptée pour un serveur, par contre pour un ordinateur de bureau, il
est souvent souhaitable d'avoir des versions plus récentes des logiciels. C'est
possible, grâce à \textbf{Debian testing}, une version de Debian qui est en
\textit{rolling release}. Le désavantage c'est que ça peut être un peu moins
stable, mais les Debianeux vous diront qu'il n'y a pratiquement jamais de
problèmes (ceci dit, c'est en ayant des problèmes qu'on apprend le plus~!). 
% et que ubuntu est basée sur la unstable :p

\bigskip Le gestionnaire de paquets est \textbf{dpkg}, avec APT comme
sur-couche. Les paquets sont des fichiers *.deb. Le programme Synaptic est un
outil avec interface graphique pour manipuler les paquets. En ligne de commande,
\texttt{aptitude} ou \texttt{apt-get} est généralement utilisé, mais pour
certaines choses il faut utiliser d'autres commandes, comme \texttt{apt-cache}.
Petit bonus~: essayez \texttt{aptitude moo}. {\scriptsize (Pour ceux qui lisent
en diagonale~: ça marche aussi sur Ubuntu ;)}

\bigskip L'installation n'est pas aussi aisée que celle d'Ubuntu~: une fois
Debian installée, vous devrez encore installer et configurer d'autres choses,
selon votre choix. Selon certaines informations, Ubuntu serait un ancien mot
africain voulant dire~: «~je ne sais pas configurer Debian~».

\bigskip Intéressé(e)~? Alors laissez-vous guider par la \textit{Formation
Debian GNU/Linux} d'Alexis de Lattre
(\url{http://formation-debian.via.ecp.fr/}).

%%%%%%%%%%%%%%%%%%%%%%%%%%%%%%%%%%%%%%%%%%%%%%%%%%%%%%%%%%%%%%%%%%%%%%%%%%%%%%%
\pagebreak \titre{fedora.pdf}{Distributions GNU/Linux~: Fedora}

La fée Dora est maintenant une grande fille, elle vient de fêter ses 16
bougies~! Bon, pour être exact, il y a une nouvelle version \textbf{tous les six
mois} (comme Ubuntu), ce qui lui fait environ 8 ans. Une version est maintenue
durant environ 13 mois (après, il n'y a plus de mises à jour de sécurité), ce
qui est peu comparé à Debian ou une Ubuntu LTS (\textit{Long Term Support}, 3 à
5 ans). Fedora s'adresse donc plutôt au \textit{desktop}.

\bigskip Fedora ---~bien que gouvernée de façon indépendante~--- est
principalement développée par \textbf{Red~Hat}, une des entreprises qui
contribue le plus à GNU/Linux. Fedora se veut à la pointe de la technologie
(\textit{\textbf{bleeding edge}}).

\bigskip Il faut distinguer l'\textit{upstream} (en amont) du
\textit{downstream} (en aval). Les distributions (Ubuntu, Debian, Fedora, …)
font partie du downstream, tandis que les logiciels tels que le noyau Linux, le
navigateur web Firefox, la suite bureautique LibreOffice, et ainsi de suite,
font partie de l'upstream. Un des rôles d'une distribution est de s'assurer
qu'il y ait une bonne intégration des différents logiciels (via le
\textit{packaging}).

Souvent, pour qu'il y ait une bonne intégration, certaines modifications doivent
être faites. Une telle modification s'appelle un \textit{patch}. Les
développeurs de Fedora communiquent le plus possible aux développeurs amonts
pour que ces patchs soient intégrés directement en upstream. Puisque c'est
intégré upstream, Fedora ne doit plus s'en occuper, et les autres distrib'
profitent des améliorations apportées. Il est parfois reproché à d'autres
distributions, comme Ubuntu, de ne pas suffisamment \textbf{contribuer
upstream}.

\bigskip Le gestionnaire de paquets est \textbf{RPM}. PackageKit en interface
graphique, ou \texttt{yum} et \texttt{rpm} en console. Pour une utilisation
basique, cela ne change pas grand chose par rapport à dpkg/APT. Mais yum permet
de faire des choses plus avancées qui n'existent pas chez Debian, et
inversement.

\bigskip L'installation est presque aussi simple qu'Ubuntu, ça
\textbf{fonctionne «~\textit{out-of-the-box}~»}. Pour pouvoir lire des fichiers
MP3 ou des animations Flash, c'est un poil plus compliqué, car Fedora est très
strict sur tout ce qui n'est pas libre ou pose des problèmes de brevets
logiciels. Autre petite chose, Fedora est de base \textit{hardened},
c'est-à-dire qu'il y a des mécanismes supplémentaires qui renforcent la
sécurité.

\bigskip Rendez-vous sur \url{http://www.fedora-fr.org/}~!

%%%%%%%%%%%%%%%%%%%%%%%%%%%%%%%%%%%%%%%%%%%%%%%%%%%%%%%%%%%%%%%%%%%%%%%%%%%%%%%
\pagebreak \titre{gentoo.pdf}{Distributions GNU/Linux~: Gentoo}

Dans un tout autre registre, Gentoo (prononcer «~djentou~») s'adresse clairement
aux passionnés, qui sont prêt à passer plus de temps pour l'installation, mais
qui aiment bien \textbf{comprendre ce qu'il se passe «~sous le capot~»}, et
d'avoir un plus \textbf{grand contrôle} sur leur machine. Un pré-requis est de
connaitre les bases de la ligne de commande, puisque l'installation se passe
entièrement dans le terminal. Gentoo est de type \textbf{\textit{rolling
release}}~: les mises à jour se font continuellement. Donc en théorie, une fois
que c'est installé, on est tranquille.

% pas besoin de croiser les doigts tous les six mois et/ou passer une
% demi-journée pour tout réinstaller à chaque fois…

\bigskip Pour installer un logiciel, Gentoo doit le \textbf{compiler}. En effet,
le code source (la recette) est d'abord téléchargé, et puis compilé (cuisiné)
sur notre propre machine. On appelle ça une distribution «~source~». Debian et
Fedora sont des distributions «~binaires~», car les paquets contiennent les
programmes sous forme déjà compilée. Ce système de compilation s'inspire du
système d'exploitation \textbf{FreeBSD} (libre, lui aussi, mais qui ne se base
pas sur le noyau Linux).

Compiler un logiciel, ça prend du temps, beaucoup plus de temps que d'installer
un paquet binaire\footnote{Il faut donc être patient, mais les gens heureux ne
sont pas pressés.}. Mais ça offre une bien plus grande flexibilité~! En effet,
\textbf{Portage}, le gestionnaire de paquets, permet ---~pour presque chaque
paquet (ou \textit{ebuild})~--- d'activer ou non certaines options. Exemple~: si
nous ne sommes pas intéressé par l'IPv6\footnote{Si vous ne savez pas ce qu'est
l'IPv6, c'est que vous n'en avez pas encore besoin.}, on peut désactiver
l'option de manière globale pour tous les logiciels, ce qui allège les binaires
générés et accélère légèrement le temps de démarrage de ces programmes.

\bigskip Portage permet de faire d'autres choses intéressantes, comme installer
en parallèle plusieurs versions différentes d'un même logiciel, si les
développeurs ont pris soin de les placer dans des \textit{slots} différents.

\bigskip Dès l'installation, on a énormément de choix sur ce que l'on peut
installer (mais tout est bien indiqué, il suffit de se laisser guider). Il va de
soi que l'on va essayer de n'installer que le strict nécessaire, alors qu'Ubuntu
ou Fedora installent souvent des choses dont on n'a pas besoin.

\bigskip Tout compris~? Alors fonce vers le (long mais instructif)
\textit{handbook}~!\\ \url{http://www.gentoo.org/} \label{last-page}
\end{document}
